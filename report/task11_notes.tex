\documentclass[12pt,a4paper]{scrartcl}
\usepackage[utf8]{inputenc}
\usepackage[english,russian]{babel}
\usepackage{indentfirst}
\usepackage{misccorr}
\usepackage{graphicx}
\usepackage{amsmath}
\begin{document}
\section{Общая схема постановки задачи}
\textbf{Важно: } рассматриваем только задачу классификации для простоты.
\begin{enumerate}
\item Как описывается модель : https://github.com/bahleg/tex\_slides/blob/master/ioi18/slides.pdf (слайд 7).
\item Общая идея параметризации берется из DARTS.
\item Гиперпараметры --- $l_2$-регулязирация.
\item Задачу ставим как двухуровневую задачу оптимизации (параметры оптимизируем по обучающей выборке, структуру и гиперпараметры --- по валидационной). Приблизительно списать можно, например, отсюда\\: https://arxiv.org/pdf/1602.02355.pdf (формула 1 на странице 3).
\end{enumerate}

\section{Общие обозначения}
\begin{itemize}
\item Обучающая выборка: $\mathfrak{D}^{\text{train}} = \{\mathbf{x}_i, y_i\}, \quad i=1,\dots,m^{\text{train}}$.
\item Валидационная выборка: $\mathfrak{D}^{\text{valid}} = \{\mathbf{x}_i, y_i\}, \quad i=1,\dots,m^{\text{valid}}$.    
\item Объекты: $[\mathbf{x}_1, \dots, \mathbf{x}_m] = \mathbf{X}, \quad \mathbf{x} \in \mathbb{R}^n.$
\item Метки объектов: $[y_1, \dots, y_m] \in \mathbf{y}, \quad y \in \{1, \dots Z\}.$
\item Количество классов: $Z$.
\item Модель: $\mathbf{f}(\mathbf{x}, \mathbf{W}).$
\item Подмодель: $\mathbf{f}_v (\mathbf{x}, \mathbf{w}_v)$.
\item Параметры модели: $\mathbf{W}$.
\item Ребра графа: $E$.
\item Вершины графа: $V$.
\item Базовые функции для ребра (i,j): $\mathbf{g}^{i,j}, \quad |\mathbf{g}^{i,j}| = K^{i,j}.$
\item Веса каждой базовой функции: $\boldsymbol{\gamma}^{i,j}$.
\item Структура модели: $\boldsymbol{\Gamma}$.
\item Регуляризационное слагаемое: $\mathbf{A}$.
\item Функция потерь на обучении: $L$
\item Функция потерь на валидации: $Q$.
\end{itemize}
\end{document}
