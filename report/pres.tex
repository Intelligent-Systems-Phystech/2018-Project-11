\documentclass[10pt]{beamer}
\usepackage[utf8]{inputenc}
\usepackage[english,russian]{babel}
\usepackage{amsmath,mathrsfs,mathtext}
\usepackage{graphicx, epsfig}
\usepackage{caption}
\usepackage{subfig}
\usepackage{amsmath}

\usepackage{multicol}

\usepackage{tikz}

\DeclareMathOperator*{\argmin}{arg\,min}
\DeclareMathOperator*{\argmax}{arg\,max}

\makeatletter
\let\@@magyar@captionfix\relax
\makeatother

\fontsize{10}{15}

\usetheme{Warsaw}
\usecolortheme{sidebartab}
\definecolor{beamer@blendedblue}{RGB}{31,96,49}

%----------------------------------------------------------------------------------------------------------
\title[\hbox to 56mm{Нейросети оптимальной сложности  \hfill\insertframenumber\,/\,\inserttotalframenumber}]
{Автоматическое построение нейросети оптимальной сложности }
\author[В.\,О. Маркин, А.\,Г. Забазнов, Н.\,А. Горян, С.\,Е. Губанов, С.\,К. Таранов]{\large \\Маркин Валерий, Забазнов Антон, Горян Николай, Сергей Губанов, Сергей Таранов, Товкес Артём, Улитин Александр, Криницкий Константин}
\institute{\large
Московский физико-технический институт}

\date{\footnotesize{10 декабря, 2018г.}}
%----------------------------------------------------------------------------------------------------------
\begin{document}
%----------------------------------------------------------------------------------------------------------
\begin{frame}
\titlepage
\end{frame}
%-----------------------------------------------------------------------------------------------------
\begin{frame}{Цель работы}

{\bf Иследуется}\\
\quad
	 Задача выбора структуры нейронной сети.\\
	~\\

{\bf Требуется}\\
\quad
	Найти нейросеть оптимальной сложности.\\
	~\\

{\bf Проблемы}\\
	\begin{itemize}
		\item Большое количество параметров,
		\item Высокая вычислительная сложность оптимизации,
		\item Невозможность использования эвристических и переборных алгоритмов выбора струкутры модели
	\end{itemize}

\end{frame}
%----------------------------------------------------------------------------------------------------------

\begin{frame}{Литература}

	\begin{itemize}
		\item \textit{LeCun Y., Denker J. , Solla S.}\\ Optimal Brain Damage~// Advances in Neural Information Processing Systems, 1989. Vol. 2. P. 598--605.
		\item	\textit{Graves A.}\\ Practical Variational Inference for Neural Networks~// Advances in Neural Information Processing Systems, 2011. P. 2348--2356.
	\end{itemize}
	
	\begin{itemize}
		\item \textit{Bishop C.}\\ Pattern Recognition and Machine Learning. --- Berlin: Springer, 2006. 758 p.
		\item \textit{Grunwald P. A}\\  Tutorial introduction to the minimum description length principle. 2005.		
	\end{itemize}
	
\end{frame}
%----------------------------------------------------------------------------------------------------------

\begin{frame}{Постановка задачи}

\[
\mathfrak{D}^{\text{train}} = \{\mathbf{x}_i, y_i\}, \quad i=1,\dots,m^{\text{train}},
\]
\[
\mathfrak{D}^{\text{valid}} = \{\mathbf{x}_i, y_i\}, \quad i=1,\dots,m^{\text{valid}},
\]
 где $\mathbf{x}_i\in\mathbf{X}\subset\mathbb{R}^{\text{n}},\quad y_i\in\mathbf{Y}\subset\mathbb{R}.$\\
~\\
$y\in\mathbf{Y}= \{1,\dots,Z\}$, где $Z$ - количество классов.\\
~\\
Модель задаётся ориентированным графом $\mathbf{G=(V,E)}$\\
~\\
$\mathbf{g}^{i,j} $--- базовые функции ребра $(i, j) $ c весами $\boldsymbol{\gamma}^{i,j}$\\
~\\
Требуется построить такую модель $\mathbf{f}$ c параметрами $\mathbf{W}\in\mathbb{R}^\text{n}$:
\[
\mathbf{f}(\mathbf{x}, \mathbf{W})= \{ \mathbf{f}_i(\mathbf{x}, \mathbf{w}_i)\}_{i=1}^\mathbf{|V|}
\]
где $\mathbf{f_i(x, w_i)}$ - подмодель c параметрами $\mathbf{w}_i$ задаётся как:
\[
\mathbf{f}_i(\mathbf{x}, \mathbf{w}_i)\ = \sum_{j\in adj(i)} \left\langle {\boldsymbol{\gamma}^{i,j}, \mathbf{g}^{i,j}} \right\rangle \mathbf{f}_j(\mathbf{x}, \mathbf{w}_j)\
\].


\end{frame}

%----------------------------------------------------------------------------------------------------------

\begin{frame}{Постановка задачи}

Функция потерь на обучении $L$ и функция потерь на валидации $Q$ задаются как:
\[
L (\mathbf{W}, \mathbf{A}, \boldsymbol{\Gamma})= \log p(\mathbf{Y}^\text{train}|\mathbf{X}^\text{train}, \mathbf{W}, \boldsymbol{\Gamma}) + \boldsymbol{e}^{\mathbf{A}}||\mathbf{W}||^2,
\]
\[
Q (\mathbf{W}, \boldsymbol{\Gamma})= \log p(\mathbf{Y}^\text{valid}|\mathbf{X}^\text{valid}, \mathbf{W}, \boldsymbol{\Gamma}) + \lambda p(\boldsymbol{\Gamma}),
\]
где $\mathbf{A}$ и $\lambda$ --- регуляризационные слагаемые, $p(\boldsymbol{\Gamma})$ - произведение всех произведение вероятностей всех $\boldsymbol{\gamma}^{i,j} \in \boldsymbol{\Gamma}$. \\
~\\
Требуется решить задачу двухуровневой оптимизации, оптимизируя параметры модели по обучающей выборке, а структуру модели по валидационной: 
\[
\mathbf{W}^*( \boldsymbol{\Gamma}) = \argmin_{\mathbf{W}}
L (\mathbf{W}, \boldsymbol{\Gamma})\]

\[
\boldsymbol{\Gamma}^*, \mathbf{A}^* = \min_{\boldsymbol{\Gamma}, \mathbf{A}} Q (\mathbf{W}^*( \boldsymbol{\Gamma}), \boldsymbol{\Gamma})
\]



\end{frame}

%----------------------------------------------------------------------------------------------------------



\begin{frame}{Вывод}



\end{frame}
%----------------------------------------------------------------------------------------------------------




\end{document} 