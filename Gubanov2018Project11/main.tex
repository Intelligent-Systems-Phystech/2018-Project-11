\documentclass[12pt,twoside]{article}

\usepackage{jmlda}

\begin{document}
\title
    {Автоматическое построение нейросети оптимальной сложности}
    %или лучше автоматизированное?
\author
    {Губанов$^1$~С.Е.} % основной список авторов, выводимый в оглавление
\email
    {sergey.gubanov@phystech.edu}
\organization
    {$^1$Московский физико-технический институт}
\abstract
	{Работа посвящена оптимизации структуры нейронной сети. Классическое обучение нейронной сети предполагает фиксированную структуру и гиперпараметры, заданные при проектировании, в то время как только непосредственные параметры обучаются. Зачастую оказывается, что количество параметров сильно завышено, а структура неоптимальна, что приводит к невысокой скорости обучения и переобучению. В данной работе предлагается новый метод обучения, который позволяет учитывать особенности задачи, подстраивая структуру и гиперпараметры в процессе. Результатом становится устойчивая модель, дающая приемлемое качество результатов при меньшей вычислительной сложности.
		
\bigskip
\textbf{Ключевые слова}: \emph {нейронные сети, оптимизация гиперпараметров, вычислительный граф, прореживание нейронной сети, устойчивость}.

}
\maketitle

%\section{1 Введение}

%\section{2 Постановка задачи}

%\section{6 Заключение}

\end{document}
