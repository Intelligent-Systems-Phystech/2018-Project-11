\documentclass[12pt,twoside]{article}


\usepackage{graphicx}
\usepackage{caption}
\usepackage{footnote}
\usepackage{jmlda}

\begin{document}
\title
    {Автоматическое построение нейронной сети оптимальной сложности}
\author
    {Товкес А.\,А.} % основной список авторов, выводимый в оглавление
\email
    {tovkes.aa@phystech.edu}
\abstract
	{Работа посвящена задаче выбора оптимальной по сложности модели нейросети. Нейросеть представляется в виде вычислительного графа, где ребрам соответствуют базовые функции, а вершинам —- промежуточные представления выборки под действием этих функций. Параметры сети разделяются на непосредственно параметры модели, которые определяют за итоговое качество классификации; гиперпараметры, определяющие процесс обучения и предотвращение переобучения; структурные параметры, определяющие непосредственно структуру модели 
    Для решения задачи оптимизации предлагается проводить релаксацию структуры нейросети. Рассмотрено изменение характеристик нейросети при колебании структурных параметров. Для анализа качества представленного алгоритма проводятся эксперименты на выборках Boston, MNIST и CIFAR-10.

	

\bigskip
\textbf{Ключевые слова}: \emph {нейронные сети, оптимизация гиперпараметров, оптимальная структура нейронной сети}.

}
\maketitle
\end{document}
