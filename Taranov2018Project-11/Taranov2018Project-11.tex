\documentclass[12pt, twoside]{article}

\usepackage{jmlda}

\begin{document}

\title{\textsc{Автоматическое определение релевантности параметров нейросети}}

\author
    {Таранов$^1$~С.К. В. Бахтеев$^1$~О.\,Ю.  Стрижов$^{1,2}$~В.\,В.} 

\email
    {taranov.sk@phystech.edu; bakhteev@phystech.edu; strijov@phystech.edu}
    
\organization
    {$^1$Московский физико-технический институт\par
    $^2$Вычислительный центр им. А.~А. Дородницына ФИЦ ИУ РАН}

\abstract
	{В данной работе исследуется выбор оптимальной структуры нейронной сети.  Модели нейронных сетей зачастую содержат большое число обучаемых параметров, предполагается, что их число можно снизить с сохранением точности прогноза. Предлагается метод, корректирующий модель в процессе обучения на основе идеи представления сети в виде графа, рёбра которого являются примитивными функциями, а вершины - промежуточными представленими выборки, полученные под действием этих функций. Для решения задачи оптимизации предлагается проводить релаксацию структуры нейросети, так чтобы соответствующая модель удовлетворяла требованиям точности для данной задачи. Также проводятся численные эксперименты на выборках данных Boston, MNIST, CIFAR-10.

\bigskip
\textbf{Ключевые слова}: \emph {нейронные сети, оптимизация гиперпараметров, релаксация графа}.

}

\maketitle

\end{document}
