\documentclass[12pt,twoside]{article}


\usepackage{graphicx}
\usepackage{caption}
\usepackage{footnote}
\usepackage{jmlda}


\begin{document}

\title
    {Автоматическое построение нейросети оптимальной сложности}
\author
    {Горян$^1$~Н.\,А. Бахтеев$^1$~О.\,Ю.  Стрижов$^2$~В.\,В.} % основной список авторов, выводимый в оглавление

\organization
    {$^1$Московский физико-технический институт\par
    $^2$Вычислительный центр им. А.~А. Дородницына ФИЦ ИУ РАН}

\email
    {goryan.na@phystech.edu; bakhteev@phystech.edu; strijov@phystech.edu}    


    

\abstract
	{Работа посвящена выбору оптимальной модели нейронной сети. Нейронная сеть рассматривается как вычислительный граф, рёбра которого --- примитивные функции, которые являются функциями активации, а вершины --- промежуточные представления выборки. Предполагается, что структуру нейронной сети можно упростить без значимой потери качества классификации. Для определения нужной структуры графа предлагается выбрать гиперпараметры и структурные параметры. Для их определения будут использоваться алгоритмы DARTS и случайный поиск. Для анализа качества представленного алгоритма проводятся эксперименты на выборках Boston, MNIST и CIFAR-10.

\bigskip

\textbf{Ключевые слова}: \emph {нейронные сети, оптимизация гиперпараметров, прореживание нейронной сети, оптимальная структура нейронной сети, вариационный вывод}.
}


\maketitle

\end{document}
